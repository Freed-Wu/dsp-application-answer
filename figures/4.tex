\documentclass[tikz]{standalone}
\usepackage{circuitikz}
\begin{document}

\tikzset{block/.style =
  dipchip,
  num pins = 10,
  hide numbers,
  no topmark,
  external pins width = 0,
}
\ctikzset{multipoles/dipchip/width = 2.2}
\begin{tikzpicture}[transform shape, node distance = 4cm, font = \tiny]
  \node[block] (l) {F28335};
  \node[block, right of = l] (r) {IS61LV51216};
  \draw (l.bpin 6) node[left] {XRD$_n$} --  (r.bpin 5) node[right] {RD$_n$};
  \draw (l.bpin 7) node[left] {XWE$_n$} --  (r.bpin 4) node[right] {WE$_n$};
  \draw (l.bpin 8) node[left] {XZCS6$_n$} --  (r.bpin 3) node[right] {CE$_n$};
  \draw[very thick] (l.bpin 9) node[left] {XD\lbrack 15:0\rbrack} --  (r.bpin 2) node[right] {D\lbrack 15:0\rbrack};
  \draw[very thick] (l.bpin 10) node[left] {XA\lbrack 18:0\rbrack} --  (r.bpin 1) node[right] {A\lbrack 18:0\rbrack};
  \draw (r.bpin 7) node[left] {BLE$_n$} -| ++ (1, -1) node[ground] (gnd) {};
  \draw (r.bpin 6) node[left] (bhe) {BHE$_n$} -- (bhe -| gnd);
\end{tikzpicture}

\end{document}
